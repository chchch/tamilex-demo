\documentclass[14pt]{extarticle}
\usepackage{polyglossia,fontspec,xunicode}
\usepackage[normalem]{ulem}
\usepackage[noend,noeledsec,noledgroup]{reledmac}
\usepackage[margin=1in]{geometry}

\arrangementX[A]{paragraph}
\arrangementX[B]{paragraph}
\renewcommand*{\thefootnoteB}{\Roman{footnoteB}}
\arrangementX[C]{paragraph}
\renewcommand*{\thefootnoteC}{\roman{footnoteC}}

\newcommand*{\caesura}{\linebreak}

\Xarrangement[A]{paragraph}
\Xnotenumfont[A]{\bfseries}
\Xlemmafont[A]{\bfseries}

\setdefaultlanguage{sanskrit}
\setotherlanguage{english}

\newfontfamily\devanagarifont[
    Path = ./fonts/brill-typeface/,
    UprightFont = brill-roman,
    BoldFont = brill-bold,
    ItalicFont = brill-italic,
    BoldItalicFont = brill-bold-italic,
    Language = Default,
    Script = Latin,
    Extension = .ttf]{Brill}
\newfontfamily\devafont[
    Path = ./fonts/brill-typeface/,
    UprightFont = brill-roman,
    BoldFont = brill-bold,
    ItalicFont = brill-italic,
    BoldItalicFont = brill-bold-italic,
    Extension = .ttf]{Brill}

\usepackage[Devanagari,DevanagariExtended]{ucharclasses}

\makeatletter
\setTransitionsFor{Devanagari}%
 {\let\curfamily\f@family\let\curshape\f@shape\let\curseries\f@series\devafont}
 {\fontfamily{\curfamily}\fontshape{\curshape}\fontseries{\curseries}\selectfont}
\makeatother

\makeatletter
\setTransitionsFor{DevanagariExtended}%
 {\let\curfamily\f@family\let\curshape\f@shape\let\curseries\f@series\devafont}
 {\fontfamily{\curfamily}\fontshape{\curshape}\fontseries{\curseries}\selectfont}
\makeatother

\begin{document}
    \raggedright

\lineation{page}
\begingroup
\beginnumbering

\pstart
 śārīraṃ nāma prathamam adhikaraṇaṃ prathamo 'dhyāyaḥ 
\pend


\setstanzaindents{2,2,2,2,2,2,2,2,2,2,2,2,2,2,2}
\stanza[\smallskip]

 \large \textsc{[Sahadeva]} &
\large yā śāstrarūpeṇa vicitrarūpair anuṣṭhitā viśvam idaṃ vicitram | &
\large ādyantahīnā khalu nirmimīte sarasvatī tāṃ praṇamāmi devīṃ || 1 || &
\large veditā sarvaśāśtrāṇāṃ bhaṭṭo ’bhūn mukulābhidhaḥ | &
\large labdhvā kutaścid ādarśāṃ bhraṣṭāmnāyaṃ sumuddhṛtam || 2 || &
\large kāvyālaṅkāraśāstraṃ yat tenaitad vāmanoditam | &
\large asūyā tan na kartavyā viśeṣālokibhiḥ kvacit || 3 || &
\large ākarṇya bhavatas tasmād dayitasya vidhīyate | &
\large vivṛtiḥ sahadevena vāmanīyasya samprati || 4 || &
\large vispaṣṭārtham idam prāyaḥ sūtraṃ vṛttisvarūpakam | &
\large vivṛttis tan mayaitasmin saṃkṣepeṇa viracyate || 5 || \&



\setstanzaindents{2,2,2,2,2,2,2,2,2,2,2,2,2,2,2}
\stanza[\smallskip]

 \large \textsc{[Vāmana]} &
\large praṇamya paramaṃ jyotir vāmanena kavipriyā | &
\large kāvyālaṅkārasūtrāṇāṃ sveṣāṃ vṛttir vidhīyate || \&



\pstart

            \textsc{[Sahadeva]} praṇamyetyādinā iṣṭadevatāṃ namaskṛtya sūtrāṇi vyācaṣṭe | kavipriyeti | kavīn prīṇātīti kaviprīḥ tena kavipriyā vāmanena | kavipriyā kavivallabhā vā vṛttiḥ || 
\pend


\setstanzaindents{2,2,2,2,2,2,2,2,2,2,2,2,2,2,2}
\stanza[\smallskip]

 \large kāvyaṃ grāhyam alaṅkārāt || 1.1.1 || \&



\pstart
 \textsc{[Vāmana]}kāvyaṃ khalu grāhyam upādeyaṃ bhavati alaṅkārāt | kāvyaśabdo ’yaṃ guṇālaṅkārasaṃskṛtayoḥ śabdārthayor vartate | bhaktyā tu śabdārthamātravacano 'tra gṛhyate || 
\pend


\pstart
 \textsc{[Sahadeva]} \textbf{kāvyaṃ} \textbf{grāhyam} \textbf{alaṅkārād} ityādikayā pañcasūtryā sambandhābhidheyaprayojanāni pratipādayitum upakramate | doṣā heyatvena guṇālāṅkārāś copādeyatvenābhidheyāḥ, śāstram abhidhāyakam, yad vakṣyati: śāstratas te iti | śāstrasya doṣādīnāṃ cābhidhānābhidheyalakṣaṇaḥ sambandho ’rthād uktaḥ, doṣādiparijñānaṃ ca prayojanam | tasyāpi prayojanaṃ doṣahānena guṇālaṅkārādānena sundarakāvyaniṣpattiḥ | tasyāpi prayojanaṃ dṛṣṭādṛṣṭaṃ ca, yad vakṣyati svakaṇṭhena — kāvyaṃ sad dṛṣṭādrṣṭārthaṃ prītikīrtihetutvāt | atas trīṇy api sambandhābhidheyaprayojanāni pratipāditāni | adhyāyānte tu prayojanasthāpānety upasaṃhāraḥ prayojanasya prādhānyāt | prayojanaṃ hi sarvatra pravṛttihetuḥ, yad uktam: “prayojanam anuddiśya na mando ’pi pravartate” iti | 
\pend


\pstart
 nanu guṇālaṅkārajanitaśobhau śabdārthau kāvyaṃ na tu kevalau | tat kiṃ bhedena pratipādyate kāvyaṃ grāhyam alaṅkārād ity āśaṅkyāha — kāvyaśabdo ’yam iti | mukhyārthaparityāgena lākṣaṇikam arthaṃ bhajata iti bhaktir lakṣaṇā | sā cātrābhidheyasambandhāt | guṇālaṅkārasaṃskṛtaśabdārthalakṣaṇo hi samudāyaḥ kāvyaśabdābhidheyaḥ | tatsambaddhaś ca śabdārthamātralakṣaṇo ’vayavaḥ | tad uktam — 


    abhidheyena saṃbandhāt sādṛśyāt samavāyataḥ | 

\smallskip
vaiparītyāt kriyāyogāl lakṣaṇā pañcadhā matā || \\
iti | prayojanaṃ cātra pṛthaguddhṛtaguṇālaṅkārasvarūpavyutpādanam | 
\pend


\pstart
 \textsc{[Vāmana]}ko ’sāv alaṅkāra ity ata āha 
\pend


\setstanzaindents{2,2,2,2,2,2,2,2,2,2,2,2,2,2,2}
\stanza[\smallskip]

 \large saundaryam alaṅkāraḥ || 1.1.2 || \&



\pstart
 \textsc{[Vāmana]}alaṅkṛtir alaṅkāraḥ | karaṇavyutpattyā punar alaṅkāraśabdo ’yam upamādiṣu vartate || 
\pend



\endnumbering
\endgroup
\end{document}
